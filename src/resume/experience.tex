\cvsection{Experience}
\begin{cventries}
  \cventry
    {Product Owner}
    {SevOne, Inc}
    {Newark, De}
    {Aug. 2016 - PRESENT}
    {
      \begin{cvitems}
        \item{\textbf{Product Owner}: Responsible for the identification and prioritization of features for the brand new product; \href{https://www.sevone.com/sevone-data-insight}{SevOne Data Insight}, which is a realtime insight and analytics platform built with cutting-edge technology.  Critical in the adoption of the \href{http://reactjs.org}{ReactJS} UI library and \href{http://graphql.org}{GraphQL} Query Language for maximum flexibility of the reporting framework. Responsible for the design and prototyping of a \textbf{geospatial focused} mapping and topology reporting 'widget' using \href{http://www.leaflet.org}{Leaflet}, \href{http://www.openstreetmap.org}{Open Street Map Data}, and \href{http://www.keylines.com}{Keylines}.  This 'widget' allowed for network mapping and realtime monitoring of complex data networks with advanced capability for realtime indicator information and alerting covering networks from within a data center to covering the globe. Was also instrumental in designing a self-hosted solution for geospatial data for customers unable or unwilling to use 3rd party hosted mapping data.}
      \end{cvitems}
    }

  \cventry
    {Project Manager - Lead Developer}
    {geographIT}
    {Lancaster, Pa}
    {Jan. 2012 - Aug. 2016}
    {
      \begin{cvitems}
        \item{\textbf{System/Application Architect}: Responsible for working with GIS lead analysts to design the best solution for complicated project/product requirements. Collected complex customer requirements and determined best technologies/architecture to fill customer needs within a fixed budget.  Projects often involved integrating many pre-existing systems with multiple data repositories into a single system architecture for numerous users with critical use cases. Successfully created numerous systems within highly critical/sensitive infrastructures that have been maintained with minimal effort for years with high customer satisfaction. }
        \bigskip
        \item{\textbf{Lead Developer}: Responsible for setting internal code standards for a small team of developers.  Instituted developer team internal standards for tools and design patterns as well as direction of technological research and development.  Moved the team from \texttt{subversion} to \texttt{git} and \texttt{gitflow} workflow to add QA/QC oversight to projects.  Instructed development team on using newer technologies such as \texttt{node} and \texttt{express}, \texttt{meteor}, \texttt{ember}, and \texttt{react}.  Trained development team via code reviews to improve code reuse and extensibility by instituting unit testing and improving code via design patterns. }
        \bigskip
        \item{\textbf{Project Manager}: Responsible for managing all aspects of a development project: determining system/project requirements, designing the system/project scope and design, creating technical specification documents and proposals, assigning and managing developers during application lifecycle, managing project budget and schedule.  Successfully managed several projects using traditional waterfall and agile project paradigms.  Responsible for managing several projects/products simultaneously.  Successfully managed client relationships by maintaining accurate schedules, enforcing high code quality, minimizing bugs, and remaining under-budget. Insured high customer satisfaction by a fine attention to UI/UX with a focus on attention to usability details.  Used extensive experience with non-savvy users to create intuitive workflows to aid in project adoption and use. }
      \end{cvitems}
    }
  \cventry
    {Senior Developer}
    {geographIT}
    {Lancaster, Pa}
    {Aug. 2009 - Dec. 2011}
    {
      \begin{cvitems}
        \item{\textbf{Incident Management System, City of Philadelphia}: Lead software development for a new ArcGIS for Server web based incident management system and custom executive dashboard that summarizes incident trends and status for the Mayor’s Office, city department heads, and utilities.  The system is used to create, track, coordinate response, and close out incidents involving street closings and downed utility lines during major natural or man-made disasters which require activation of the EOC as the City’s command and control center.  Participated in system requirements gathering meetings, prepared software requirements specifications, designed application architecture, estimated the software development budget, and provided technical guidance to two other software developers.  Responsible for the development of a custom configurable JavaScript dashboard that summarizes metrics for incident trends and incident responses directly from the incident database and displays the metrics as a panel of trend graphs and charts.}
        \bigskip
        \item{\textbf{Urban Forestry Asset and Inventory Management System, Washington D.C.}: Responsible for preparing the system requirements specification and developing and implementing a custom urban forestry asset management application to track tree planting and maintenance at the tree farm, inventorying potted trees acquired from suppliers, and tree planting and maintenance activities in the Washington DC metropolitan area.  The application architecture includes ArcGIS for Server 10.3 published REST services, Apple iPads and iPhones running Apple iOS 9, and the ArcGIS for Apple iOS SDK 10.3.  Tree planting, inventory, and maintenance activities are entered from the field in offline mode and synchronize edits to ArcGIS for Server services using cellular broadband or WiFi hotspots.}
        \bigskip
        \item{\textbf{Journey Management Application, Lancaster, Pa}: Lead developer for a Journey Management solution developed for a large energy company involved in Marcellus Shale natural gas production.  This document retrieval web-GIS application enables interactive map and database searches to locate predefined Acrobat pdf map documents of preplanned routes to/from the well site and intersection drawings.  The configurable web application enables changing application settings by Rettew system administrators.  The application was developed within four weeks from date of contract using an agile approach to application development with weekly demonstrations of progress to the client.  The Amazon EC2 cloud hosted application uses ArcGIS Server 10.x map and data services and was developed using Esri’s JavaScript API for ArcGIS Server.}
        \bigskip
        \item{\textbf{SWEEP Mobile Ticking System, City of Philadelphia}: Lead developer for Philadelphia Department of Streets, Sanitation Division Streets \& Walkways Education and Enforcement Program (SWEEP) mobile ticketing application. This application uses GPS to show the user’s location on a map. The application allows writing of SWEEP violations on site, while capturing photographs of violations, capturing officer signatures, and using bluetooth printing to deliver violations to violators. The SWEEP Application uses RFID reader technology to scan RFID tags attached to refuse dumpsters for display of licensing information. It also uses Microsoft’s SQL Server synchronization framework for Microsoft’s SQL Server compact edition via Microsoft’s Internet Information Services (IIS). The application has supporting Microsoft SQL Service Integration Services packages that extract, transform, and load data from an Oracle database containing licensing information and a nightly export process that uploads tickets to the ticket payment collection department and emails tickets to violators.}
        \bigskip
        \item{\textbf{Philadelphia Mobile Pole and LED Asset Tracking Program, City of Philadelphia}: Lead developer for Philadelphia mobile pole and LED asset tracking program to use with the city’s conversion to LED street lights. This application uses GPS to show user’s location on the map. The application allows addition, editing, and deletion of map features in the field for asset management. The Asset Tracking Program uses a barcode reader to track LED installation and removal efficiently. It uses ESRI’s mobile synchronization framework to synchronize changes across devices and publishes edits on a central server.}
      \end{cvitems}
    }
  \cventry
    {Developer}
    {AMQ Software}
    {Lancaster, Pa}
    {Dec. 2008 - Aug. 2009}
    {
      \begin{cvitems}
        \item {\textbf{Opti-Mogul Route Optimization Dispatch Software, Lancaster, Pa}: Lead developer for this project for Opti-Mogul. This mobile application uses external GPS to show a user’s location on a map. The mobile application receives the order of deliveries, delivery information, and routing information to delivery, and sends the driver time clock information and delivery payment/status information back using Windows Communication Foundation services with Microsoft Message Queue. The dispatch software extracts, transforms, and exports and updates data from Eagle Business Management Software using nHibernate. This application allows editing of new fuel stations, trucks, truck compartments, and manual stop assignment/ordering while synchronizing any changes to mobile clients.}
      \end{cvitems}
    }
  \cventry
    {Developer}
    {Swift Signal}
    {Lancaster,  Pa}
    {Jun. 2008 - Dec. 2008}
    {
      \begin{cvitems}
        \item {\textbf{Windows .Net Desktop Software, Lancaster, Pa}: Lead Developer of Windows .Net desktop software for AMQ Software, including project Epsilon, a fully customizable Point of Sale system generalized enough to handle all retail markets. Designed and developed codename Epsilon from creation to current version. Implemented multi-station/store database environment using nHibernate. Implemented data synchronization and multi-store searching using WCF. Implemented Credit/Debit/Gift/Check processing using Mercury Payment system.}
      \end{cvitems}
    }
  \cventry
    {Developer}
    {Iron Compass Maps}
    {Lancaster, Pa}
    {Jul. 2004 - Jun. 2008}
    {
      \begin{cvitems}
        \item {\textbf{On-Scene Xplorer Application Suite, Lancaster, Pa}: Lead Developer of Iron Compass Map Company’s flagship product, On-Scene Xplorer, a mapping and pre-incident planning application for the emergency services industry, as well as designing and implementing tools and applications for internal and external company use. Designed and developed On-Scene Xplorer using ESRI’s MapObjects from creation to 2.5 release. Supervised and contributed to porting codebase to C\# and .Net Framework 3.5. Designed and implemented a PocketPC application to capture location and attributes of fire hydrants to be imported into On-Scene Xplorer data format. Designed and implemented various tools to improve coworker productivity, including but not limited to: Data verification tool used to guarantee accurate user data before distribution; various custom cartography tools; application for automatically creating Nullsoft Install System installation scripts; downloaded and merged aerial photography; ArcGIS plugin to ease the creating of building footprints from aerial photography.}
      \end{cvitems}
    }
\end{cventries}
